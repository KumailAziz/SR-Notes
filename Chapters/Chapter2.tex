%Chapter 2

\section{Foundations of Special Relativity}

\subsection{Relativity of Simultaneity}

\subsection{Time Dilation}
\textcolor{purple}{This subsection is from \textit{Time in motion} to \textit{Time dilation: experimental evidence}.}

\subsubsection{Moving mirrors}
*What is a clock?
It is any physical system that undergoes cyclical repetitive motion in a uniform way. 

For example, a light clock is a two mirrors facing each other and a light bulb going up and down (tic toc). This light clock reveals
how motion affects passage of time. \textcolor{blue}{Draw the figure in page 6, on Kumail's notes. It shows how that motion slows down time.} 

Does the observer in motion feel ro recognize that his/her clock is running slow?
The answer is No, he/she will NOT feel the \textit{uniform} motion.  

Another question, why don't we notice time running slow in everyday life?
The answer is that because the difference is very very small to notice.
\subsubsection{Derivation}

\subsection{Length Contraction}

\subsection{Paradoxes of Simultaneity}

\subsection{Lorentz Transformations}
\subsubsection{Properties}
\subsubsection{Addition of velocities}