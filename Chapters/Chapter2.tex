%Chapter 2

\section{Foundations of Special Relativity}

\subsection{Relativity of Simultaneity}

\subsection{Time Dilation}
\textcolor{purple}{This subsection is from \textit{Time in motion} to \textit{Time dilation: experimental evidence}.}

\subsubsection{Moving mirrors}
*What is a clock?
It is any physical system that undergoes cyclical repetitive motion in a uniform way. 

For example, a light clock is a two mirrors facing each other and a light bulb going up and down (tic toc). This light clock reveals
how motion affects passage of time. \textcolor{blue}{Draw the figure in page 6, on Kumail's notes. It shows how that motion slows down time.} 

Does the observer in motion feel ro recognize that his/her clock is running slow?
The answer is No, he/she will NOT feel the \textit{uniform} motion.  

Another question, why don't we notice time running slow in everyday life?
The answer is that because the difference is very very small to notice.
\subsubsection{Derivation}

\subsection{Length Contraction}

\subsection{Paradoxes of Simultaneity}

\subsection{Lorentz Transformations}
\subsubsection{Properties}
\subsubsection{Addition of velocities}
\textcolor{purple}{This subsubsection is from \textit{Combining velocities} to \textit{Combining velocities: Example in 3D}.}

The expectation of adding velocities: \\
If you run towards(away) a light beam with speed v, then the relative light speed to you with is $c+v(c-v)$. And this is wrong!

It will be like this:
\[ w-v \rightarrow \frac{w-v}{1-vw/c^2}\]
\[ w+v \rightarrow \frac{w+v}{1+vw/c^2}\]

The proof is: the cheeta example.. See it later..

Platform:- \\
The cheeta $(t, wt)$; You $(t, vt)$ \\
So, what is the speed of cheeta from your perspective?

\[ t' = \gamma (t - vx/c^2) = \gamma (t - v/c^2 \bullet wt) = \gamma t(t - vw/c^2) \]
\[ x' = \gamma (x - vt) = \gamma (wt - vt) = \gamma t(w - v) \]

Thus, the cheeta speed from your perspective is $\frac{x'}{t'}$: 
\[ = \frac{\gamma t(w - v)}{\gamma t(t - vw/c^2)} = \frac{(w - v)}{(t - vw/c^2)} \]

What we have done is for one dimensino, what if the moving object moves in 3D? For instance, if the train frame has a velocity v in direction x,
and the object moves in the platform with \textbf{w}, $(w_x, w_y, w_z)$. Our goal now is to find the velocity of the object from the perspective of 
the train?

Platform: $(t, w_x t, w_y t, w_z t)$
Train: $(t', x', y', z') = (\gamma (t-vw_x t/c^2), \gamma (w_x t - vt), w_y t, w_z t)$

\[ w'_x = \frac{\gamma (w_x t - vt)}{\gamma (t-vw_x t/c^2)} = \frac{(w_x - v)}{(t - vw_x/c^2)}\]
\[ w'_y = \frac{w_y t}{\gamma (t-vw_y t/c^2)} = \frac{w_y}{\gamma (t-vw_y/c^2)} \]
\[ w'_z = \frac{w_z t}{\gamma (t-vw_z t/c^2)} = \frac{w_z}{\gamma (t-vw_z/c^2)} \]

\textbf{Example: 1D} \\
Graice runs by George at $0.8c$ and yells "tag, your it". He then runs after her at $0.7c$. From his perspective, how quickly is she getting away?
\textcolor{blue}{Draw a schematic figure.}

Solution:\\
Graice's velocity: $w=0.8c$ \\
George's velocity: $v=0.7c$ \\

\[ \frac{w-v}{1-wv/c^2} = \frac{0.8c - 0.7c}{1-0.56} = 0.23c\]

\textbf{Example: 3D} \\
Graice turns a sharp corner, and now runs at $0.8c$ north, while George still runs at $0.7c$ due east. From George's perspective, what is Graice's
velocity now?
\textcolor{blue}{Draw a schematic figure.}

Solution: \\
Graice's velocity: $\overrightarrow{w}=(0,0.8c,0)$ \\
George's velocity: $\overrightarrow{v}=(0.7c,0,0)$ \\

\[ (\frac{(w_x - v)}{(t - vw_x/c^2)}, \frac{w_y}{\gamma (t-vw_y/c^2)}, \frac{w_z}{\gamma (t-vw_z/c^2)})\]
\[ = (\frac{0-0.7c}{1-0},\frac{0.8c}{1.4*(1-0)},0)\]
\[= (-0.7c, 0.57c, 0)\]